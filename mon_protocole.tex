\documentclass[a4paper,11pt]{article}
\usepackage[utf8]{inputenc}
\usepackage[french]{babel}
%opening
\title{Cahier des Charges Memoire de Licence}
\author{GOUDALO Jospy}

\begin{document}

\maketitle
\vspace{4cm}
\tableofcontents
\newpage

\section*{Th\`eme}
  Prototype d'un syst\`eme de paiement \'electronique des factures d'\'electricit\'e de la SBEE.

\section*{Introduction}
  Le Bénin s’inscrit depuis quelques années dans une politique de restructuration du secteur numérique pour y insuffler une nouvelle dynamique. Plusieurs projets et programmes sont donc nés de cette volonté et constituent une feuille de route pour les acteurs au cœur de cette restructuration. 
  \\Dans l'optique de contribuer au developpement de ce secteur et surtout de faciliter la vie \`a la population b\'eninoise, nous avous pens\'e \`a mettre en place une plateforme permettant le paiement \`a distance des factures d'\'electricit\'e de la SBEE.
  
\section{Problématique}
  Il n'est pas rare de constater qu'apr\`es de longues heures d'attente au guichet de la SBEE, l'on vous dise: ``D\'esol\'e monsieur nous avons un probl\`eme de connexion. Veuillez repasser demain.''. Il faudra donc repasser plus tard pour solder la facture alors que nous n'avons pas forc\'ement assez de temps pour cela. \\
  Nous sommes aussi parfois confrontr\'es \`a l'oubli des factures non pay\'ees. Cependant quand vous n'etes pas \`a jour apr\`es un d\'elai d'environ un mois, un agent peut passer \`a tout moment couper le courant et vous devez payer des p\'enalit\'es.\\
  Notre travail consistera donc \`a mettre en place une application web s\'ecuris\'ee permettant le paiement des factures depuis un telephone portable ou un ordinateur connect\'e \`a Internet.

 \section{Hypothèse de travail}
     \begin{itemize}
       \item Les longues queues aux guichets
       \item Les probl\`emes emp\^echant les validations des paiements aux guichets
       \item L'oubli de factures impay\'ees
     \end{itemize}

\section{Objectifs de l'étude}
  \subsection{Objectif général}
    \begin{itemize}
      \item permettre \`a tout client de la SBEE disposant d'Internet de payer ses factures en restant dans les temps
      \item s\'ecuriser les transactions financi\`eres relatives \`a ces paiements
      \item faciliter les paiements
    \end{itemize}

  \subsection{Objectifs sp\'ecifiques}
    \begin{itemize}
      \item assurer la disponibilit\'e compl\`ete de la plateforme afin que les consultations et paiements puissent se faire \`a n'importe quel moment.
      \item garantir la s\'ecurit\'e de toutes les transactions financi\`eres effectu\'ees via la plateforme
      \item construire une historique afin d'avoir une trace des factures pay\'ees et impay\'ees
      \item rappeler aux personnes utilisant la plateforme qu'ils ont des impay\'es (via des SMS et/ou des emails)
    \end{itemize}

\section{Comp\'etences requises}
   \begin{itemize}
     \item Bases solides sur l'e-paiement
     \item Developpement d'application Web / API / Int\'egration
     \item Sécurité des applications
     \item Systèmes cryptographiques de sécurité
     \item Conception et administration des bases de données
     \item Configuration et administration de serveurs web
   \end{itemize}


\section{Livrables}
  A la fin de notre \'etude, nous livrerons les \'el\'ements suivants: 
  \begin{itemize}
    \item le document complet de mémoire
    \item les sources de l'application développée
  \end{itemize}

\vspace*{1cm}
  
  \flushright
    Fait \`a Abomey-Calavi le, 13 Juin 2018\\
\vspace*{1.5cm}
    GOUDALO Jospy
    
\vspace*{1cm}  
  \noindent \textbf{Encadreur / Ecole:} \hfill \textbf{Encadreur / Stage:} \\
\vspace*{1.5cm}
  \noindent M. EZIN Eug\`ene \hfill M. METINHOUE Camille
  
\end{document}